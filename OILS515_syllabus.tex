% Options for packages loaded elsewhere
\PassOptionsToPackage{unicode=true}{hyperref}
\PassOptionsToPackage{hyphens}{url}
%
\documentclass[
]{article}
\usepackage{lmodern}
\usepackage{amssymb,amsmath}
\usepackage{ifxetex,ifluatex}
\ifnum 0\ifxetex 1\fi\ifluatex 1\fi=0 % if pdftex
  \usepackage[T1]{fontenc}
  \usepackage[utf8]{inputenc}
  \usepackage{textcomp} % provides euro and other symbols
\else % if luatex or xelatex
  \usepackage{unicode-math}
  \defaultfontfeatures{Scale=MatchLowercase}
  \defaultfontfeatures[\rmfamily]{Ligatures=TeX,Scale=1}
\fi
% Use upquote if available, for straight quotes in verbatim environments
\IfFileExists{upquote.sty}{\usepackage{upquote}}{}
\IfFileExists{microtype.sty}{% use microtype if available
  \usepackage[]{microtype}
  \UseMicrotypeSet[protrusion]{basicmath} % disable protrusion for tt fonts
}{}
\makeatletter
\@ifundefined{KOMAClassName}{% if non-KOMA class
  \IfFileExists{parskip.sty}{%
    \usepackage{parskip}
  }{% else
    \setlength{\parindent}{0pt}
    \setlength{\parskip}{6pt plus 2pt minus 1pt}}
}{% if KOMA class
  \KOMAoptions{parskip=half}}
\makeatother
\usepackage{xcolor}
\IfFileExists{xurl.sty}{\usepackage{xurl}}{} % add URL line breaks if available
\IfFileExists{bookmark.sty}{\usepackage{bookmark}}{\usepackage{hyperref}}
\hypersetup{
  pdftitle={Organization, Information, and Learning Sciences (OILS) 515 / Geography and Environmental Studies (GEOG) 522 - Introduction to Spatial Data Management (3 Credit Hours)- Syllabus},
  pdfauthor={Karl Benedict, Associate Professor - College of University Libraries and Learning Sciences},
  hidelinks,
}
\urlstyle{same} % disable monospaced font for URLs
\usepackage{longtable,booktabs}
% Allow footnotes in longtable head/foot
\IfFileExists{footnotehyper.sty}{\usepackage{footnotehyper}}{\usepackage{footnote}}
\makesavenoteenv{longtable}
\setlength{\emergencystretch}{3em} % prevent overfull lines
\providecommand{\tightlist}{%
  \setlength{\itemsep}{0pt}\setlength{\parskip}{0pt}}
\setcounter{secnumdepth}{-\maxdimen} % remove section numbering
% Redefines (sub)paragraphs to behave more like sections
\ifx\paragraph\undefined\else
  \let\oldparagraph\paragraph
  \renewcommand{\paragraph}[1]{\oldparagraph{#1}\mbox{}}
\fi
\ifx\subparagraph\undefined\else
  \let\oldsubparagraph\subparagraph
  \renewcommand{\subparagraph}[1]{\oldsubparagraph{#1}\mbox{}}
\fi

% Set default figure placement to htbp
\makeatletter
\def\fps@figure{htbp}
\makeatother


\title{Organization, Information, and Learning Sciences (OILS) 515 / Geography
and Environmental Studies (GEOG) 522 - Introduction to Spatial Data
Management (3 Credit Hours)- Syllabus}
\author{Karl Benedict, Associate Professor - College of University Libraries and
Learning Sciences}
\date{Fall 2019}

\begin{document}
\maketitle

Many data products are inherently spatial. Obviously, spatial data
include data collection locations, but many other data may also be
considered spatial: locations in space that documents pertain to and
locations of historic or literary events illustrating just a few. While
maps are a familiar product derived from spatial data, there is
significant understanding of the underlying data -- the processes to
which it has been subjected, the actual values within the data, the
originator of the data, any limitations in the appropriate use of the
data, and the nature of the dataset itself (format, scale, coordinate
system, units) -- that is required before it can be productively used
for research or applications. This course is designed to provide
graduate-level students with the necessary skills and knowledge to meet
this challenge through hands-on work in \emph{discovering},
\emph{creating}, \emph{managing}, \emph{using}, \emph{documenting} and
\emph{sharing} spatial data. After completing this course students will
be prepared to develop a plan for the management of their spatial data,
locate and evaluate data sources that they need for their research,
create and structure data that they collect for maximum value both
during and after their research project, and document their data
throughout their research projects, maximizing the impact of their
research and the value of the data they generate and share with other
researchers.

\hypertarget{course-instructor}{%
\section{Course Instructor}\label{course-instructor}}

Karl Benedict

\begin{itemize}
\tightlist
\item
  Associate Professor, Director of Research Data Services, and
  Information Technology, College of University Libraries and Learning
  Sciences
\item
  Affiliated Faculty, Department of Geography
\end{itemize}

\emph{Email}: \href{mailto:kbene@unm.edu}{\nolinkurl{kbene@unm.edu}}

\emph{Phone}: (505) 277-5256

\emph{Office Hours}: Online or in-person, by appointment.\\
Appointment Calendar:
\url{https://libcal.unm.edu/appointments/karlbenedict}\\
If a needed appointment time isn't available through the calendar send
me an email to arrange for alternative times.

\emph{Office Location}:\\
Centennial Science and Engineering Library - CSEL L173

\hypertarget{course-description-and-objectives}{%
\section{Course Description and
Objectives}\label{course-description-and-objectives}}

An understanding of core spatial data concepts and principles is
increasingly important in the current world of collaborative, spatially
enabled research and applications. We are no longer working in a vacuum
as individual researchers that only need to understand and use the data
that we create and use in our separate research projects. Successful
research depends upon being able to integrate data generated by others
with our own and by extension being able to share our data with others,
both during our research projects and also for posterity (and
increasingly to meet the requirements of funding agencies, publishers,
and our peers). This class will focus on the following aspects of
spatial data management that relate to this need for effective
integration, use, collaboration and sharing:

\begin{itemize}
\tightlist
\item
  The \emph{Research} and \emph{Data Lifecycles}
\item
  Types of spatial data
\item
  Spatial database design and management
\item
  Working with and managing gridded data
\item
  Spatial data documentation standards and practices
\item
  Data management planning
\item
  Ethical, legal and privacy issues as they relate to spatial data
\item
  Strategies for sharing spatial data
\end{itemize}

Upon completion of the course students will have knowledge and skills in
the following areas:

\begin{itemize}
\tightlist
\item
  Locating and evaluating spatial data based upon knowledge of formats,
  content models and documentation standards
\item
  Structuring data (both in terms of format selection and content) from
  a variety of sources to enable integrated research
\item
  Evaluate data products to determine which elements of a dataset might
  raise ethical, legal or privacy issues if released or shared with
  others
\item
  Documenting data as an ongoing process throughout the research cycle
\item
  Producing machine- and human-readable documentation for data to
  support discovery, understanding, and use of data that they produce
\item
  The options and strategies for effectively sharing their spatial data
  with collaborators and transitioning those data into long-term
  preservation systems
\end{itemize}

Detailed weekly objectives are available in the separate Goals and
Objectives document.
(\href{http://karlbenedict.com/OILS515/goalsAndObjectives/goalsAndObjectives.html}{Web
page}) \textbar{}
(\href{http://karlbenedict.com/OILS515/goalsAndObjectives/goalsAndObjectives.pdf}{PDF})

\hypertarget{course-format}{%
\section{Course Format}\label{course-format}}

The course is structured as a combination of short
lectures/demonstrations that set the stage for the technical topics
covered in the readings, hands-on work with data and data documentation,
and data management planning exercises. While offered as an online
course, several online web conferences (collaboratory sessions) are
required as part of the class participation. The times of these sessions
will be established during the first week of class based on the
availability of times for the students and class instructor.

\hypertarget{readings}{%
\section{Readings}\label{readings}}

All of the course readings are available through online resources
available through UNM Library subscriptions and databases or directly
through open access materials. A number of the readings for the course
are collected in a shared \emph{Safari Books Online}
\href{https://learning.oreilly.com/playlists/96eb9098-8389-4d2a-959d-2064e0114b6b}{playlist}.
You will need to create an account with Safari Books Online the first
time you access the collection. From then on you will login with the
account information you used when you created the account. The specific
readings for each week are provided in the separate Goals and Objectives
document.
(\href{http://karlbenedict.com/OILS515/goalsAndObjectives/goalsAndObjectives.html}{Web
page}) \textbar{}
(\href{http://karlbenedict.com/OILS515/goalsAndObjectives/goalsAndObjectives.pdf}{PDF})

\hypertarget{evaluation-and-grading}{%
\section{Evaluation and Grading}\label{evaluation-and-grading}}

Course grades will be based on a combination of participation in live
and online discussions and peer-review, the smaller assignments (listed
under the ``Assignment'' column in the class calendar), and the
semester-long class project. The grade for the class will be weighted
according to the following breakdown:

\begin{itemize}
\tightlist
\item
  Class Participation: 20\% (50 pts)

  \begin{itemize}
  \tightlist
  \item
    Attending required collaboratory sessions in weeks 1, 4, 8, 12, and
    16 (10 pts)
  \item
    Data Management Plan peer review (20 pts)
  \item
    Final Project peer review (20 pts)
  \end{itemize}
\item
  Small Assignments: 40\% (100 pts)

  \begin{itemize}
  \tightlist
  \item
    Literature Review (33 pts)
  \item
    Data Review (33 pts)
  \item
    Data Management Plan (34 pts)
  \end{itemize}
\item
  Class Project: 40\% (100 pts)

  \begin{itemize}
  \tightlist
  \item
    Step 1 - Project plan (10 pts)
  \item
    Step 2 - Dataset descriptions (10 pts)
  \item
    Step 3 - Submission of project materials for peer review (5 pts)
  \item
    Step 4 - Submission of final datasets and documentation (50 pts) and
    presentation (25 pts)
  \end{itemize}
\end{itemize}

\textbf{Late Assignments} - in the absence of \textbf{documented}
circumstances beyond your control that prevent your timely submission of
all assignments you will lose 10\% of the potential points for the
assignment \textbf{per day} until the assignment is submitted.

Based upon a total of 250 points for the class, the following breakdown
of earned grades will be in effect:

\begin{longtable}[]{@{}lll@{}}
\toprule
\begin{minipage}[b]{0.20\columnwidth}\raggedright
Percent\strut
\end{minipage} & \begin{minipage}[b]{0.14\columnwidth}\raggedright
Points\strut
\end{minipage} & \begin{minipage}[b]{0.16\columnwidth}\raggedright
Letter Grade\strut
\end{minipage}\tabularnewline
\midrule
\endhead
\begin{minipage}[t]{0.20\columnwidth}\raggedright
96.67 - 100.00\strut
\end{minipage} & \begin{minipage}[t]{0.14\columnwidth}\raggedright
242 - 250\strut
\end{minipage} & \begin{minipage}[t]{0.16\columnwidth}\raggedright
A+\strut
\end{minipage}\tabularnewline
\begin{minipage}[t]{0.20\columnwidth}\raggedright
93.33 - 96.66\strut
\end{minipage} & \begin{minipage}[t]{0.14\columnwidth}\raggedright
233 - 241\strut
\end{minipage} & \begin{minipage}[t]{0.16\columnwidth}\raggedright
A\strut
\end{minipage}\tabularnewline
\begin{minipage}[t]{0.20\columnwidth}\raggedright
90.00 - 93.32\strut
\end{minipage} & \begin{minipage}[t]{0.14\columnwidth}\raggedright
225 - 232\strut
\end{minipage} & \begin{minipage}[t]{0.16\columnwidth}\raggedright
A-\strut
\end{minipage}\tabularnewline
\begin{minipage}[t]{0.20\columnwidth}\raggedright
86.67 - 89.99\strut
\end{minipage} & \begin{minipage}[t]{0.14\columnwidth}\raggedright
217 - 224\strut
\end{minipage} & \begin{minipage}[t]{0.16\columnwidth}\raggedright
B+\strut
\end{minipage}\tabularnewline
\begin{minipage}[t]{0.20\columnwidth}\raggedright
83.33 - 86.66\strut
\end{minipage} & \begin{minipage}[t]{0.14\columnwidth}\raggedright
208 - 216\strut
\end{minipage} & \begin{minipage}[t]{0.16\columnwidth}\raggedright
B+\strut
\end{minipage}\tabularnewline
\begin{minipage}[t]{0.20\columnwidth}\raggedright
80.00 - 83.32\strut
\end{minipage} & \begin{minipage}[t]{0.14\columnwidth}\raggedright
200 - 207\strut
\end{minipage} & \begin{minipage}[t]{0.16\columnwidth}\raggedright
B-\strut
\end{minipage}\tabularnewline
\begin{minipage}[t]{0.20\columnwidth}\raggedright
76.67 - 79.99\strut
\end{minipage} & \begin{minipage}[t]{0.14\columnwidth}\raggedright
192 - 199\strut
\end{minipage} & \begin{minipage}[t]{0.16\columnwidth}\raggedright
C+\strut
\end{minipage}\tabularnewline
\begin{minipage}[t]{0.20\columnwidth}\raggedright
73.33 - 76.66\strut
\end{minipage} & \begin{minipage}[t]{0.14\columnwidth}\raggedright
183 - 191\strut
\end{minipage} & \begin{minipage}[t]{0.16\columnwidth}\raggedright
C\strut
\end{minipage}\tabularnewline
\begin{minipage}[t]{0.20\columnwidth}\raggedright
00.00 - 73.32\strut
\end{minipage} & \begin{minipage}[t]{0.14\columnwidth}\raggedright
000 - 182\strut
\end{minipage} & \begin{minipage}[t]{0.16\columnwidth}\raggedright
F\strut
\end{minipage}\tabularnewline
\bottomrule
\end{longtable}

As this is a graduate course grades of C-, D+, D, and D- are not allowed
for graduate students. As a result these grades are not included in the
grading scheme.

While students are encouraged to collaborate in their work on the
project and homework assignments, submitted work must be original and
written and submitted by each individual student.

Please refer to the \href{http://pathfinder.unm.edu/}{Pathfinder} for
detailed student conduct policies, and in particular the following
\href{http://pathfinder.unm.edu/campus-policies/academic-dishonesty.html}{Policy
on Academic Dishonesty}.

\begin{quote}
Each student is expected to maintain the highest standards of honesty
and integrity in academic and professional matters. The University
reserves the right to take disciplinary action, up to and including
dismissal, against any student who is found guilty of academic
dishonesty or otherwise fails to meet the standards. Any student judged
to have engaged in academic dishonesty in course work may receive a
reduced or failing grade for the work in question and/or for the course.
\end{quote}

\begin{quote}
Academic dishonesty includes, but is not limited to, dishonesty in
quizzes, tests, or assignments; claiming credit for work not done or
done by others; hindering the academic work of other students;
misrepresenting academic or professional qualifications within or
without the University; and nondisclosure or misrepresentation in
filling out applications or other University records.
\end{quote}

\hypertarget{technical-requirements}{%
\section{Technical Requirements}\label{technical-requirements}}

\hypertarget{skills}{%
\subsection{Skills}\label{skills}}

\begin{itemize}
\tightlist
\item
  Use UNM Learn (help documentation located in ``How to Use Learn'' link
  on left course menu, and also at
  http://online.unm.edu/help/learn/students/ ).
\item
  Use email -- including attaching files, opening files, downloading
  attachments
\item
  Copy and paste within applications including Microsoft Office
\item
  Open a hyperlink (click on a hyperlink to get to a website or online
  resource)
\item
  Use Microsoft Office applications

  \begin{itemize}
  \tightlist
  \item
    Create, download, update, save and upload MS Word documents
  \item
    Create, download, update, save and upload MS PowerPoint
    presentations
  \end{itemize}
\item
  Use the in-course web conferencing tool (Collaborate Web Conferencing
  software)
\item
  Download and install an application or plug in -- required for
  participating in web conferencing sessions and running other
  applications required for successful work in the course (see below for
  a list of software requirements)
\end{itemize}

\hypertarget{software}{%
\subsection{Software}\label{software}}

\begin{itemize}
\tightlist
\item
  Recent Windows, Mac or Linux Operating System
\item
  GIS - Quantum GIS (QGIS) - \url{http://www.qgis.org/}
\item
  Spatial Database - SpatiaLite (with Rasterlite support) - installed as
  part of QGIS
\item
  Python (possible, based upon interest)
\item
  Microsoft Office products are available free for all UNM students
  (more information on the UNM IT Software Distribution and Downloads
  page: http://it.unm.edu/software/index.html)
\end{itemize}

\hypertarget{hardware}{%
\subsection{Hardware}\label{hardware}}

\begin{itemize}
\tightlist
\item
  A high speed Internet connection is highly recommended.
\item
  Supported browsers include: Internet Explorer, Firefox, and Safari.
  Detailed Supported Browsers and Operating Systems:
  http://online.unm.edu/help/learn/students/
\item
  Any computer capable of running a recently updated web browser should
  be sufficient to access your online course. However, bear in mind that
  processor speed, amount of RAM and Internet connection speed can
  greatly affect performance. Many locations offer free high-speed
  Internet access including UNM's Computer Pods. The size and complexity
  of the datasets you select for your class project may require
  additional memory and storage beyond the baseline required to access
  the online learning environment.
\end{itemize}

\hypertarget{web-conferencing}{%
\subsection{Web Conferencing}\label{web-conferencing}}

Web conferencing will be used in this course during the following times
and dates:

\begin{itemize}
\tightlist
\item
  August 21, 2019. 5:00-7:00 pm
\item
  September 11, 2019. 5:00-7:00 pm
\item
  October 9, 2019. 5:00-7:00 pm
\item
  November 6, 2019. 5:00-7:00 pm
\item
  December 4, 2019. 5:00-7:00 pm
\end{itemize}

For the online sessions, you will need:

\begin{itemize}
\tightlist
\item
  A USB headset with microphone. Headsets are widely available at stores
  that sell electronics, at the UNM Bookstore or online.
\item
  A high-speed internet connection is highly recommended for these
  sessions. A wireless Internet connection may be used if successfully
  tested for audio quality prior to web conferencing.
\item
  For UNM Web Conference Technical Help: (505) 277-0857
\end{itemize}

\textbf{Tracking Course Activity} - UNM Learn automatically records all
students' activities including: your first and last access to the
course, the pages you have accessed, the number of discussion messages
you have read and sent, web conferencing, discussion text, and posted
discussion topics. This data can be accessed by the instructor to
evaluate class participation and to identify students having difficulty

\hypertarget{weekly-schedule}{%
\section{Weekly Schedule}\label{weekly-schedule}}

\begin{longtable}[]{@{}llllll@{}}
\toprule
\begin{minipage}[b]{0.05\columnwidth}\raggedright
Week\strut
\end{minipage} & \begin{minipage}[b]{0.11\columnwidth}\raggedright
Dates\strut
\end{minipage} & \begin{minipage}[b]{0.18\columnwidth}\raggedright
Topic\strut
\end{minipage} & \begin{minipage}[b]{0.18\columnwidth}\raggedright
Collaboratory\strut
\end{minipage} & \begin{minipage}[b]{0.14\columnwidth}\raggedright
Assignment\strut
\end{minipage} & \begin{minipage}[b]{0.17\columnwidth}\raggedright
Project Milestones\strut
\end{minipage}\tabularnewline
\midrule
\endhead
\begin{minipage}[t]{0.05\columnwidth}\raggedright
1\strut
\end{minipage} & \begin{minipage}[t]{0.11\columnwidth}\raggedright
Aug 19-25\strut
\end{minipage} & \begin{minipage}[t]{0.18\columnwidth}\raggedright
Course Overview - Introduction to the Data Lifecycle\strut
\end{minipage} & \begin{minipage}[t]{0.18\columnwidth}\raggedright
Class Introduction\strut
\end{minipage} & \begin{minipage}[t]{0.14\columnwidth}\raggedright
-\strut
\end{minipage} & \begin{minipage}[t]{0.17\columnwidth}\raggedright
-\strut
\end{minipage}\tabularnewline
\begin{minipage}[t]{0.05\columnwidth}\raggedright
2\strut
\end{minipage} & \begin{minipage}[t]{0.11\columnwidth}\raggedright
Aug 26 - Sep 01\strut
\end{minipage} & \begin{minipage}[t]{0.18\columnwidth}\raggedright
Types of Spatial Data - Vector\strut
\end{minipage} & \begin{minipage}[t]{0.18\columnwidth}\raggedright
-\strut
\end{minipage} & \begin{minipage}[t]{0.14\columnwidth}\raggedright
Start literature review\strut
\end{minipage} & \begin{minipage}[t]{0.17\columnwidth}\raggedright
Define data management focus for term\strut
\end{minipage}\tabularnewline
\begin{minipage}[t]{0.05\columnwidth}\raggedright
3\strut
\end{minipage} & \begin{minipage}[t]{0.11\columnwidth}\raggedright
Sep 02-08\strut
\end{minipage} & \begin{minipage}[t]{0.18\columnwidth}\raggedright
Types of Spatial Data - Raster\strut
\end{minipage} & \begin{minipage}[t]{0.18\columnwidth}\raggedright
-\strut
\end{minipage} & \begin{minipage}[t]{0.14\columnwidth}\raggedright
-\strut
\end{minipage} & \begin{minipage}[t]{0.17\columnwidth}\raggedright
-\strut
\end{minipage}\tabularnewline
\begin{minipage}[t]{0.05\columnwidth}\raggedright
4\strut
\end{minipage} & \begin{minipage}[t]{0.11\columnwidth}\raggedright
Sep 09\strut
\end{minipage} & \begin{minipage}[t]{0.18\columnwidth}\raggedright
Database design I\strut
\end{minipage} & \begin{minipage}[t]{0.18\columnwidth}\raggedright
Present literature review results\strut
\end{minipage} & \begin{minipage}[t]{0.14\columnwidth}\raggedright
Start data review for documentation, usability and understanding\strut
\end{minipage} & \begin{minipage}[t]{0.17\columnwidth}\raggedright
-\strut
\end{minipage}\tabularnewline
\begin{minipage}[t]{0.05\columnwidth}\raggedright
5\strut
\end{minipage} & \begin{minipage}[t]{0.11\columnwidth}\raggedright
Sep 16 - 22\strut
\end{minipage} & \begin{minipage}[t]{0.18\columnwidth}\raggedright
Database design II\strut
\end{minipage} & \begin{minipage}[t]{0.18\columnwidth}\raggedright
-\strut
\end{minipage} & \begin{minipage}[t]{0.14\columnwidth}\raggedright
-\strut
\end{minipage} & \begin{minipage}[t]{0.17\columnwidth}\raggedright
-\strut
\end{minipage}\tabularnewline
\begin{minipage}[t]{0.05\columnwidth}\raggedright
6\strut
\end{minipage} & \begin{minipage}[t]{0.11\columnwidth}\raggedright
Sep 23 - 29\strut
\end{minipage} & \begin{minipage}[t]{0.18\columnwidth}\raggedright
Geodatabase design\strut
\end{minipage} & \begin{minipage}[t]{0.18\columnwidth}\raggedright
-\strut
\end{minipage} & \begin{minipage}[t]{0.14\columnwidth}\raggedright
-\strut
\end{minipage} & \begin{minipage}[t]{0.17\columnwidth}\raggedright
-\strut
\end{minipage}\tabularnewline
\begin{minipage}[t]{0.05\columnwidth}\raggedright
7\strut
\end{minipage} & \begin{minipage}[t]{0.11\columnwidth}\raggedright
Sep 30 - Oct 06\strut
\end{minipage} & \begin{minipage}[t]{0.18\columnwidth}\raggedright
Managing raster data\strut
\end{minipage} & \begin{minipage}[t]{0.18\columnwidth}\raggedright
-\strut
\end{minipage} & \begin{minipage}[t]{0.14\columnwidth}\raggedright
-\strut
\end{minipage} & \begin{minipage}[t]{0.17\columnwidth}\raggedright
-\strut
\end{minipage}\tabularnewline
\begin{minipage}[t]{0.05\columnwidth}\raggedright
8\strut
\end{minipage} & \begin{minipage}[t]{0.11\columnwidth}\raggedright
Oct 07 - 13\strut
\end{minipage} & \begin{minipage}[t]{0.18\columnwidth}\raggedright
Data formats for Analysis and Archiving\strut
\end{minipage} & \begin{minipage}[t]{0.18\columnwidth}\raggedright
Presentations of data review\strut
\end{minipage} & \begin{minipage}[t]{0.14\columnwidth}\raggedright
-\strut
\end{minipage} & \begin{minipage}[t]{0.17\columnwidth}\raggedright
Enumerate specific data (\textgreater= three datasets) to be used in the
project\strut
\end{minipage}\tabularnewline
\begin{minipage}[t]{0.05\columnwidth}\raggedright
9\strut
\end{minipage} & \begin{minipage}[t]{0.11\columnwidth}\raggedright
Oct 14 - 20\strut
\end{minipage} & \begin{minipage}[t]{0.18\columnwidth}\raggedright
Documenting data - the interview\strut
\end{minipage} & \begin{minipage}[t]{0.18\columnwidth}\raggedright
-\strut
\end{minipage} & \begin{minipage}[t]{0.14\columnwidth}\raggedright
-\strut
\end{minipage} & \begin{minipage}[t]{0.17\columnwidth}\raggedright
Create initial data\strut
\end{minipage}\tabularnewline
\begin{minipage}[t]{0.05\columnwidth}\raggedright
10\strut
\end{minipage} & \begin{minipage}[t]{0.11\columnwidth}\raggedright
Oct 21 - 27\strut
\end{minipage} & \begin{minipage}[t]{0.18\columnwidth}\raggedright
XML Document creation, editing and validation\strut
\end{minipage} & \begin{minipage}[t]{0.18\columnwidth}\raggedright
-\strut
\end{minipage} & \begin{minipage}[t]{0.14\columnwidth}\raggedright
-\strut
\end{minipage} & \begin{minipage}[t]{0.17\columnwidth}\raggedright
-\strut
\end{minipage}\tabularnewline
\begin{minipage}[t]{0.05\columnwidth}\raggedright
11\strut
\end{minipage} & \begin{minipage}[t]{0.11\columnwidth}\raggedright
Oct 28 - Nov 03\strut
\end{minipage} & \begin{minipage}[t]{0.18\columnwidth}\raggedright
Metadata Standards - FGDC\strut
\end{minipage} & \begin{minipage}[t]{0.18\columnwidth}\raggedright
-\strut
\end{minipage} & \begin{minipage}[t]{0.14\columnwidth}\raggedright
-\strut
\end{minipage} & \begin{minipage}[t]{0.17\columnwidth}\raggedright
Document Data\strut
\end{minipage}\tabularnewline
\begin{minipage}[t]{0.05\columnwidth}\raggedright
12\strut
\end{minipage} & \begin{minipage}[t]{0.11\columnwidth}\raggedright
Nov 04 - 10\strut
\end{minipage} & \begin{minipage}[t]{0.18\columnwidth}\raggedright
Metadata Standards - ISO and Dublin Core\strut
\end{minipage} & \begin{minipage}[t]{0.18\columnwidth}\raggedright
Data management planning process Q\&A\strut
\end{minipage} & \begin{minipage}[t]{0.14\columnwidth}\raggedright
Start data management plan\strut
\end{minipage} & \begin{minipage}[t]{0.17\columnwidth}\raggedright
-\strut
\end{minipage}\tabularnewline
\begin{minipage}[t]{0.05\columnwidth}\raggedright
13\strut
\end{minipage} & \begin{minipage}[t]{0.11\columnwidth}\raggedright
Nov 11 - 17\strut
\end{minipage} & \begin{minipage}[t]{0.18\columnwidth}\raggedright
Data management planning\strut
\end{minipage} & \begin{minipage}[t]{0.18\columnwidth}\raggedright
-\strut
\end{minipage} & \begin{minipage}[t]{0.14\columnwidth}\raggedright
-\strut
\end{minipage} & \begin{minipage}[t]{0.17\columnwidth}\raggedright
-\strut
\end{minipage}\tabularnewline
\begin{minipage}[t]{0.05\columnwidth}\raggedright
14\strut
\end{minipage} & \begin{minipage}[t]{0.11\columnwidth}\raggedright
Nov 18 - 24\strut
\end{minipage} & \begin{minipage}[t]{0.18\columnwidth}\raggedright
Data management planning\strut
\end{minipage} & \begin{minipage}[t]{0.18\columnwidth}\raggedright
-\strut
\end{minipage} & \begin{minipage}[t]{0.14\columnwidth}\raggedright
-\strut
\end{minipage} & \begin{minipage}[t]{0.17\columnwidth}\raggedright
-\strut
\end{minipage}\tabularnewline
\begin{minipage}[t]{0.05\columnwidth}\raggedright
15\strut
\end{minipage} & \begin{minipage}[t]{0.11\columnwidth}\raggedright
Nov 25 - Dec 01\strut
\end{minipage} & \begin{minipage}[t]{0.18\columnwidth}\raggedright
Sharing Data\strut
\end{minipage} & \begin{minipage}[t]{0.18\columnwidth}\raggedright
-\strut
\end{minipage} & \begin{minipage}[t]{0.14\columnwidth}\raggedright
Data management plan peer review\strut
\end{minipage} & \begin{minipage}[t]{0.17\columnwidth}\raggedright
Project data and documentation peer review\strut
\end{minipage}\tabularnewline
\begin{minipage}[t]{0.05\columnwidth}\raggedright
16\strut
\end{minipage} & \begin{minipage}[t]{0.11\columnwidth}\raggedright
Dec 02 - 08\strut
\end{minipage} & \begin{minipage}[t]{0.18\columnwidth}\raggedright
Emerging concepts/ Ethical, legal and privacy issues\strut
\end{minipage} & \begin{minipage}[t]{0.18\columnwidth}\raggedright
Project Presentations\strut
\end{minipage} & \begin{minipage}[t]{0.14\columnwidth}\raggedright
-\strut
\end{minipage} & \begin{minipage}[t]{0.17\columnwidth}\raggedright
Present project results and peer review outcome\strut
\end{minipage}\tabularnewline
\bottomrule
\end{longtable}

\hypertarget{additional-notes}{%
\section{Additional Notes}\label{additional-notes}}

Our classroom and our university should always be spaces of mutual
respect, kindness, and support, without fear of discrimination,
harassment, or violence. Should you ever need assistance or have
concerns about incidents that violate this principle, please access the
resources available to you on campus, especially the LoboRESPECT
Advocacy Center and the support services listed on its website
(http://loborespect.unm.edu/). Please note that, because UNM faculty,
TAs, and GAs are considered ``responsible employees'' by the Department
of Education, any disclosure of gender discrimination (including sexual
harassment, sexual misconduct, and sexual violence) made to a faculty
member, TA, or GA must be reported by that faculty member, TA, or GA to
the university's Title IX coordinator. For more information on the
campus policy regarding sexual misconduct, please see:
\url{https://policy.unm.edu/university-policies/2000/2740.html}.

\begin{center}\rule{0.5\linewidth}{\linethickness}\end{center}

This work by {Karl Benedict} is licensed under a Creative Commons
Attribution-ShareAlike 4.0 International License.

\end{document}
